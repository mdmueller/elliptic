% This is a Homework Template for CS 1010.  
% It was modified by  Michael Carl Tschantz (mtschant) 
% who provided the ``useful infomation on latex''.

% THE ONLY THING YOU NEED TO DO IN THIS PART IS 
% TO FILL IN THE HOMEWORK NUMBER, YOUR NAME AND LOG IN BELOW
% REPLACE ``X'' WITH THE HW NUMBER, ``Your Name'' WITH YOUR NAME,
% AND ``your login'' WITH YOUR LOGIN.

\newcommand{\hwnumber}{12}
\newcommand{\yourname}{Michael Mueller}

% NOW YOU MAY SKIP DOWN TO THE PART CALLED ``YOUR DOCUMENT''.

%==============================================================================
% Formatting parameters (how the page is set up)
%==============================================================================
\newcommand{\yourcourse}{Math 695}
\newcommand{\chapter}{0}
\newcommand{\mgn}{\mathcal M_{g,n}}
\newcommand{\mgnb}{\overline{\mathcal M_{g,n}}}
\newcommand{\mgb}[1]{\overline{\mathcal M_{g,#1}}}

\documentclass[11pt]{article}           % 11pt article
\makeatletter                   % Make '@' accessible.
\pagestyle{myheadings}              % We do our own page headers.
\newcommand{\thishw}{\bf Elliptic curve question}
\def\@oddhead{\bf \thishw \hfill \yourname}
\oddsidemargin=0in              % Left margin minus 1 inch.
\evensidemargin=0in             % Same for even-numbered pages.
\textwidth=6.5in                % Text width (8.5in - margins).
\topmargin=0in                  % Top margin minus 1 inch.
\headsep=0.2in                  % Distance from header to body.
\textheight=8in                 % Body height (incl. footnotes)
\skip\footins=4ex               % Space above first footnote.
\hbadness=10000                 % No "underfull hbox" messages.
\makeatother                    % Make '@' special again.

%==============================================================================
% Packages used (packages add more commands)
%==============================================================================

\usepackage{amsmath}                % give more fonts and symbols
\usepackage{amsfonts}               % want AMS fonts
\usepackage{amssymb}
\usepackage{amsthm}
\usepackage{mathrsfs}
\usepackage{tikz-cd}
\usepackage{bbm}                % given mathbbm fonts
\usepackage[shortlabels]{enumitem}
\usepackage{relsize}
\usepackage{hyperref}

\makeatletter
\newcommand*{\relrelbarsep}{.386ex}
\newcommand*{\relrelbar}{%
  \mathrel{%
    \mathpalette\@relrelbar\relrelbarsep
  }%
}
\newcommand*{\@relrelbar}[2]{%
  \raise#2\hbox to 0pt{$\m@th#1\relbar$\hss}%
  \lower#2\hbox{$\m@th#1\relbar$}%
}
\providecommand*{\rightrightarrowsfill@}{%
  \arrowfill@\relrelbar\relrelbar\rightrightarrows
}
\providecommand*{\leftleftarrowsfill@}{%
  \arrowfill@\leftleftarrows\relrelbar\relrelbar
}
\providecommand*{\xrightrightarrows}[2][]{%
  \ext@arrow 0359\rightrightarrowsfill@{#1}{#2}%
}
\providecommand*{\xleftleftarrows}[2][]{%
  \ext@arrow 3095\leftleftarrowsfill@{#1}{#2}%
}
\makeatother

%==============================================================================
% Macros (make your own commands)
%==============================================================================

% For problem and part headers
\newcounter{problemcounter}
\newcounter{subproblemcounter}
\newcommand{\problem}{
    \addtocounter{problemcounter}{1}
    \bigskip
    \noindent {\Large Problem \hwnumber .\theproblemcounter}
    \smallskip
    \setcounter{subproblemcounter}{0}
}
\newcommand{\subproblem}{
    \addtocounter{subproblemcounter}{1}
    \smallskip
    \noindent {\bf \alph{subproblemcounter})} 
}

% Nice things
\newcommand{\set}[1]{\{#1\}}            % Set (as in \set{1,2,3})
\newcommand{\setof}[2]{\{\,{#1}|~{#2}\,\}}  % Set (as in \setof{x}{x > 0})

% Some letter symbols
\newcommand{\N}{\ensuremath{\mathbb{N}}}
\newcommand{\Z}{\ensuremath{\mathbb{Z}}}
\newcommand{\R}{\ensuremath{\mathbb{R}}}
\newcommand{\hTop}{\textbf{hTop}}
\newtheorem*{Proposition}{Proposition}
\newtheorem*{Corollary}{Corollary}
\newcommand{\Tor}{\text{Tor}}
\newcommand{\Ext}{\text{Ext}}
\newcommand{\Q}{\mathbb{Q}}
\newcommand{\F}{\mathbb{F}}
\newcommand{\C}{\mathbb{C}}
\newcommand{\CP}{\mathbb{CP}}
\newcommand{\RP}{\mathbb{RP}}
\newcommand{\Spec}{\text{Spec}}
\newcommand{\Aut}{\text{Aut}}
\newcommand{\Proj}{\text{Proj}}
\newcommand{\Mor}{\text{Mor}}
\newcommand{\codim}{\text{codim}}
\newcommand{\exer}[1]{{\bf Exercise #1} \\}
\newcommand{\Hom}{\text{Hom}}
\newcommand{\coker}{\text{coker}}
\newcommand*\simplex{\includegraphics[scale=0.017]{simplex.png}}
\newcommand{\Sch}{\textbf{Sch}}
\newcommand{\Set}{\textbf{Set}}
\renewcommand{\P}{\mathbb P}
\theoremstyle{definition}
\newtheorem*{thm}{Theorem}
\newtheorem*{prob}{Problem}
\newtheorem*{dfn}{Definition}
\newtheorem*{claim}{Claim}
\theoremstyle{definition}
\newtheorem*{lem}{Lemma}
\newtheorem*{ex}{Exercise}
\newtheorem*{eg}{Example}
\newtheorem*{note}{Note}

\usetikzlibrary{matrix,positioning,quotes}

%==============================================================================
% YOUR DOCUMENT (start here)
%==============================================================================

\begin{document}
\definecolor{myblue}{RGB}{100,180,255}
\definecolor{mygreen}{RGB}{80,160,80}
\definecolor{myred}{RGB}{200,120,100}
\centerline{\LARGE\thishw}


Let
\[
S_{\mathfrak a,\mathfrak b}(\mu)=N_{\mu,(2,1,\dots)^{b_2},(3,1,\dots)^{a_3},(3,1,\dots)^{b_3},\dots}^{\{2^{b_2}3^{b_3}\dots\}}
\]
Here $a_3,a_4,\dots$ describe unmarked ramification and $b_2,b_3,\dots$ describe marked ramification. For
simplicity, define $S_{\mathfrak a,\mathfrak b}(\mu)= 0$ if $\mu_i\leq 0$ for any $i$.

\begin{claim}
  For $n>0$, if $b_y>0$, then
  \begin{align*}
    S_{\mathfrak a,\mathfrak b}(\mu, n) &=H((n),(y,1,\dots),(n-y+1,1,\dots))S_{\mathfrak a,\mathfrak b-\{y\}}(\mu, n-y+1) \\
    &+\sum_{z\in\mathfrak a+\{2\}}H((y+z-n-2,n),(z,1,\dots),(y,1,\dots))S_{\mathfrak a-\{z\},\mathfrak b-\{y\}+\{y+z-n-2\}}(\mu)
  \end{align*}
\end{claim}

\begin{proof}
  We consider diagrams which stabilize to a genus $1$ curve containing $x_1,\dots,x_k$ and all but one
  of the other fixed points, attached to a genus $0$ tail containing $q$ (with ramification $n$) and $p$
  (with ramification $y$).
  A diagram with $x_1,\dots,x_k$ on the right side looks like:

\begin{tikzpicture}[thick,amat/.style={matrix of nodes,nodes in empty cells,
  row sep=3.2em,rounded corners,
  nodes={draw,solid,circle,minimum size=1.5cm}},
  dmat/.style={matrix of nodes,nodes in empty cells,row sep=3.2em,nodes={minimum size=1.5cm},draw=myred},
  fsnode/.style={fill=myblue},
  ssnode/.style={fill=mygreen}]

  \matrix[amat,nodes=fsnode] (mat1) {$1$\\
    $0$\\
    \vdots \\
  $0$\\};

  \matrix[dmat,left=2cm of mat1] (degrees1) {$d-\sum c_i$\\
    $c_1$\\
    \vdots \\
  $c_r$\\};

 \matrix[amat,right=7cm of mat1,nodes=ssnode] (mat2) {$0$\\
   \vdots\\
   $0$ \\
   $0$\\};
 \matrix[dmat,right=1.5cm of mat2] (degrees2) {$\mu_1$\\
   \vdots\\
   $\mu_k$ \\
 $n$\\};

 \draw  (mat1-1-1) edge["$\mu_1$"] (mat2-1-1)
 (mat1-1-1) edge["$\mu_k$"] (mat2-3-1)
 (mat1-1-1) edge["$n-\sum c_i$"] (mat2-4-1)
 (mat1-2-1) edge["$c_1$"] (mat2-4-1)
 (mat1-4-1) edge["$c_r$"] (mat2-4-1);

  % draw legs for left side
 \draw (mat1-1-1) -- +(95:1) node[anchor=south] {fixed points};

 % draw legs for right side
 \draw  (mat2-1-1) -- +(30:1) node[anchor=west] {$x_1$}
 (mat2-3-1) -- +(30:1) node[anchor=west] {$x_k$}
 (mat2-4-1) -- +(30:1) node[anchor=west] {$q$};

\end{tikzpicture}

A priori, $p$ might be on the right or left side. Now if $c_1>1$, the diagram can be amended from the left picture to the right picture without loss of generality:

  \begin{tikzpicture}[thick,amat/.style={matrix of nodes,nodes in empty cells,
  row sep=3.2em,rounded corners,
  nodes={draw,solid,circle,minimum size=1.5cm}},
  dmat/.style={matrix of nodes,nodes in empty cells,row sep=2.2em,nodes={minimum size=1.5cm},draw=myred},
  fsnode/.style={fill=myblue},
  ssnode/.style={fill=mygreen}]

  \matrix[amat,nodes=fsnode] (mat1) {$1$\\
  $0$\\};

  \matrix[amat,right=2cm of mat1,nodes=ssnode] (mat2) {$0$\\};
  
 \draw  (mat1-1-1) edge["$e$"] (mat2-1-1)
 (mat1-2-1) edge["$c_1$"] (mat2-1-1);

  % draw legs for left side
 \draw (mat1-2-1) -- +(120:1) node[anchor=south] {$c_1-1$ ramification};

 % NEW PICTURE
 
   \matrix[amat,right=2cm of mat2,nodes=fsnode] (mat3) {$1$\\
     $0$\\
     \vdots\\
   $0$\\};

  \matrix[amat,right=2cm of mat3,nodes=ssnode] (mat4) {$0$\\};
  
 \draw  (mat3-1-1) edge["$e$"] (mat4-1-1)
 (mat3-2-1) edge["$1$"] (mat4-1-1)
 (mat3-3-1) edge["\vdots"] (mat4-1-1)
 (mat3-4-1) edge["$1$"] (mat4-1-1);

 \draw (mat4-1-1) -- +(20:1) node[anchor=west] {$c_1-1$ ramification};

  % ARROW
% \draw [->] +(90:2) (mat2-1-1) -- (mat3-1-1); 

  \end{tikzpicture}

  This means we can assume that $c_1=\dots=c_r=1$ and $p$ is on the right side:

  \begin{tikzpicture}[thick,amat/.style={matrix of nodes,nodes in empty cells,
  row sep=3.2em,rounded corners,
  nodes={draw,solid,circle,minimum size=1.5cm}},
  dmat/.style={matrix of nodes,nodes in empty cells,row sep=3.2em,nodes={minimum size=1.5cm},draw=myred},
  fsnode/.style={fill=myblue},
  ssnode/.style={fill=mygreen}]

  \matrix[amat,nodes=fsnode] (mat1) {$1$\\
    $0$\\
    \vdots \\
  $0$\\};

  \matrix[dmat,left=2cm of mat1] (degrees1) {$d-r$\\
    $1$\\
    \vdots \\
  $1$\\};

 \matrix[amat,right=7cm of mat1,nodes=ssnode] (mat2) {$0$\\
   \vdots\\
   $0$ \\
   $0$\\};
 \matrix[dmat,right=1.5cm of mat2] (degrees2) {$\mu_1$\\
   \vdots\\
   $\mu_k$ \\
 $n$\\};

 \draw  (mat1-1-1) edge["$\mu_1$"] (mat2-1-1)
 (mat1-1-1) edge["$\mu_k$"] (mat2-3-1)
 (mat1-1-1) edge["$n-r$"] (mat2-4-1)
 (mat1-2-1) edge["$1$"] (mat2-4-1)
 (mat1-4-1) edge["$1$"] (mat2-4-1);

  % draw legs for left side
 \draw (mat1-1-1) -- +(95:1) node[anchor=south] {fixed points};

 % draw legs for right side
 \draw  (mat2-1-1) -- +(30:1) node[anchor=west] {$x_1$}
 (mat2-3-1) -- +(30:1) node[anchor=west] {$x_k$}
 (mat2-4-1) -- +(30:1) node[anchor=west] {$q$}
 (mat2-4-1) -- +(330:1) node[anchor=west] {$p$};

  \end{tikzpicture}

  As there is no extra ramification outside the genus $1$ component,
  the dimension reduction of $r$ should match the codimension reduction
  corresponding to the removal of $p$, which is $y-1$. So $r=y-1$ and we
  have the first term in the claim.

  Now consider the case where $x_1,\dots,x_k,q$ are on the left side.
  By converting the left picture to the right we can assume that $x_1,\dots,x_k$ lie on the genus $1$ component:

  \begin{tikzpicture}[thick,amat/.style={matrix of nodes,nodes in empty cells,
  row sep=3.2em,rounded corners,
  nodes={draw,solid,circle,minimum size=1cm}},
  dmat/.style={matrix of nodes,nodes in empty cells,row sep=3.2em,nodes={minimum size=1cm},draw=myred},
  fsnode/.style={fill=myblue},
  ssnode/.style={fill=mygreen}]

    \matrix[amat,nodes=fsnode] (mat1) {$0$\\
      $1$\\
  $0$\\};

    \matrix[amat,right=2cm of mat1,nodes=ssnode] (mat2) {$0$\\
      $0$\\};

      \matrix[dmat,left=0.5cm of mat1] (degrees1) {$\mu_1$\\
    $d-n-\mu_1$\\
        $n$\\};

            \matrix[dmat,right=0.5cm of mat2] (degrees2) {$e$\\
  $f$\\};
  
 \draw  (mat1-1-1) edge["$\mu_1$"] (mat2-1-1)
 (mat1-2-1) edge["$e-\mu_1$"] (mat2-1-1)
 (mat1-2-1) edge["$f-m$"] (mat2-2-1)
 (mat1-3-1) edge["$m$"] (mat2-2-1);

  % draw legs for left side
 \draw (mat1-1-1) -- +(120:1) node[anchor=south] {$x_1$}
 (mat1-2-1) -- +(120:1) node[anchor=south] {$x_2,\dots,x_k$}
 (mat1-3-1) -- +(120:1) node[anchor=south] {$q$};

   % draw legs for right side
 \draw (mat2-1-1) -- +(100:1) node[anchor=south] {$e$ ramif}
 (mat2-2-1) -- +(300:1) node[anchor=north] {$f$ ramif};

 % NEW PICTURE
 
   \matrix[amat,right=4cm of mat2,nodes=fsnode] (mat3) {$1$\\
     $0$\\};

   \matrix[amat,right=2cm of mat3,nodes=ssnode] (mat4) {$0$\\};

         \matrix[dmat,left=0.5cm of mat3] (degrees3) {$d-n$\\
        $n$\\};

            \matrix[dmat,right=0.5cm of mat4] (degrees4) {$e+f$\\};


 % draw legs for left side
 \draw (mat3-1-1) -- +(120:1) node[anchor=south] {$x_1,\dots,x_k$}
 (mat3-2-1) -- +(120:1) node[anchor=south] {$q$};
 
 \draw  (mat3-1-1) edge["$e+f-m$"] (mat4-1-1)
 (mat3-2-1) edge["$m$"] (mat4-1-1);

 \draw (mat4-1-1) -- +(300:1) node[anchor=north] {$e+f$ ramif};

  % ARROW
% \draw [->] +(90:2) (mat2-1-1) -- (mat3-1-1); 

  \end{tikzpicture}

  By an argument similar to earlier, we can assume the picture looks like:

    \begin{tikzpicture}[thick,amat/.style={matrix of nodes,nodes in empty cells,
  row sep=3.2em,rounded corners,
  nodes={draw,solid,circle,minimum size=1.5cm}},
  dmat/.style={matrix of nodes,nodes in empty cells,row sep=3.2em,nodes={minimum size=1.5cm},draw=myred},
  fsnode/.style={fill=myblue},
  ssnode/.style={fill=mygreen}]

  \matrix[amat,nodes=fsnode] (mat1) {$1$\\
    $0$\\};

  \matrix[dmat,left=2cm of mat1] (degrees1) {$d-n$\\
    $n$\\};

 \matrix[amat,right=3cm of mat1,nodes=ssnode] (mat2) {$0$\\
   \vdots\\
   $0$ \\
   $0$\\
   $0$\\
   \vdots\\
 $0$\\};
 \matrix[dmat,right=1.5cm of mat2] (degrees2) {$1$\\
   \vdots\\
   $1$ \\
   $e+f$\\
   $1$\\
   \vdots\\
 $1$\\};

 \draw  (mat1-1-1) edge["$1$"] (mat2-1-1)
 (mat1-1-1) edge["$1$"] (mat2-3-1)
 (mat1-1-1) edge["$e$"] (mat2-4-1)
 (mat1-2-1) edge["$f$"] (mat2-4-1)
 (mat1-2-1) edge["$1$"] (mat2-5-1)
 (mat1-2-1) edge["$1$"] (mat2-7-1);

  % draw legs for left side
 \draw (mat1-1-1) -- +(115:1) node[anchor=south] {fixed points}
 (mat1-1-1) -- +(260:1) node[anchor=north] {$x_1,\dots,x_k$}
 (mat1-2-1) -- +(140:1) node[anchor=east] {$q$};


    \end{tikzpicture}

    The top right edges are all labeled $1$ because if an edge had label $r$, the ramification $r-1$ on its corresponding
    bubble could be moved to the genus $1$ component and only increase the dimension of the genus $1$.

    There is $n-f$ ramification on the lower left component, which
    can be moved over to the right side:

    \begin{tikzpicture}[thick,amat/.style={matrix of nodes,nodes in empty cells,
  row sep=3.2em,rounded corners,
  nodes={draw,solid,circle,minimum size=1cm}},
  dmat/.style={matrix of nodes,nodes in empty cells,row sep=3.2em,nodes={minimum size=1cm},draw=myred},
  fsnode/.style={fill=myblue},
  ssnode/.style={fill=mygreen}]

  \matrix[amat,nodes=fsnode] (mat1) {$1$\\
    $0$\\};

  \matrix[dmat,left=.5cm of mat1] (degrees1) {$d-n$\\
    $n$\\};

  \matrix[amat,right=2cm of mat1,nodes=ssnode] (mat2) {$0$\\
    $0$\\
   \vdots\\
   $0$ \\};
 \matrix[dmat,right=.5cm of mat2] (degrees2) {$e+f$\\
   $1$\\
   \vdots\\
   $1$ \\};

 \draw  (mat1-1-1) edge["$e$"] (mat2-1-1)
 (mat1-2-1) edge["$f$"] (mat2-1-1)
 (mat1-2-1) edge["$1$"] (mat2-2-1)
 (mat1-2-1) edge["$1$"] (mat2-4-1);

  % draw legs for left side
 \draw (mat1-1-1) -- +(115:.8) node[anchor=south] {fixed points}
 (mat1-1-1) -- +(260:.8) node[anchor=north] {$x_1,\dots,x_k$}
 (mat1-2-1) -- +(140:.8) node[anchor=east] {$q$}
 (mat1-2-1) -- +(240:.8) node[anchor=north] {$n-f$ ramif};

   % draw legs for right side
 \draw (mat2-1-1) -- +(115:.8) node[anchor=south] {$e+f$ ramif};

 % NEW PICTURE

   \matrix[amat,right=4cm of mat2,nodes=fsnode] (mat3) {$1$\\
    $0$\\};

  \matrix[dmat,left=.5cm of mat3] (degrees3) {$d-n$\\
    $n$\\};

  \matrix[amat,right=3cm of mat3,nodes=ssnode] (mat4) {$0$\\};
 \matrix[dmat,right=.5cm of mat4] (degrees4) {$e+n$\\};

 \draw  (mat3-1-1) edge["$e$"] (mat4-1-1)
 (mat3-2-1) edge["$n$"] (mat4-1-1);

  % draw legs for left side
 \draw (mat3-1-1) -- +(115:.8) node[anchor=south] {fixed points}
 (mat3-1-1) -- +(260:.8) node[anchor=north] {$x_1,\dots,x_k$}
 (mat3-2-1) -- +(140:.8) node[anchor=east] {$q$};

   % draw legs for right side
 \draw (mat4-1-1) -- +(260:.8) node[anchor=north] {$n-f$ ramif}
 (mat4-1-1) -- +(115:.8) node[anchor=south] {$e+f$ ramif};
    \end{tikzpicture}

    So we can consider the picture:

        \begin{tikzpicture}[thick,amat/.style={matrix of nodes,nodes in empty cells,
  row sep=3.2em,rounded corners,
  nodes={draw,solid,circle,minimum size=1.5cm}},
  dmat/.style={matrix of nodes,nodes in empty cells,row sep=3.2em,nodes={minimum size=1.5cm},draw=myred},
  fsnode/.style={fill=myblue},
  ssnode/.style={fill=mygreen}]

  \matrix[amat,nodes=fsnode] (mat1) {$1$\\
    $0$\\};

  \matrix[dmat,left=2cm of mat1] (degrees1) {$d-n$\\
    $n$\\};

 \matrix[amat,right=3cm of mat1,nodes=ssnode] (mat2) {$0$\\
   \vdots\\
   $0$ \\
   $0$\\};
 \matrix[dmat,right=1.5cm of mat2] (degrees2) {$1$\\
   \vdots\\
   $1$ \\
   $e+n$\\};

 \draw  (mat1-1-1) edge["$1$"] (mat2-1-1)
 (mat1-1-1) edge["$1$"] (mat2-3-1)
 (mat1-1-1) edge["$e$"] (mat2-4-1)
 (mat1-2-1) edge["$n$"] (mat2-4-1);

  % draw legs for left side
 \draw (mat1-1-1) -- +(115:1) node[anchor=south] {fixed points}
 (mat1-1-1) -- +(260:1) node[anchor=north] {$x_1,\dots,x_k$}
 (mat1-2-1) -- +(140:1) node[anchor=east] {$q$};

   % draw legs for right side
 \draw (mat2-4-1) -- +(75:1) node[anchor=south] {$p$}
 (mat2-4-1) -- +(265:1) node[anchor=north] {$n+e-y+1$ ramif};


 \end{tikzpicture}

        The codimension of the genus $1$ component has increased by
        $n+e-y$, since the degree is reduced by $n$ and the fixed point
        of ramification index $y$ is replaced by a fixed point of ramification
        index $e$. Either $n+e-y=0$, in which case the bottom right component
        has only one simple additional ramification (and this suffices), or
        the bottom right component has ramification $n+e-y+1=z-1$
        (i.e., $z=n+e-y+2$) and the point of ramification index $z$ balances
        out the loss of an equivalent $z$ on the genus $1$ component. In this
        case for arbitrary $z$ (of which the previous is a special case with $z=2$),
        we have $e=y+z-n-2$.
        
\end{proof}

This claim is not quite stated correctly, because there is some subtlety with
the Hurwitz numbers (I think they should be ``non-orbifold'' Hurwitz counts).
While I can show that the Hurwitz number in the $z=2$ case is always $1$, I haven't
yet computed the others.

\end{document}
