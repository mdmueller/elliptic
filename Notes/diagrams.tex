% This is a Homework Template for CS 1010.  
% It was modified by  Michael Carl Tschantz (mtschant) 
% who provided the ``useful infomation on latex''.

% THE ONLY THING YOU NEED TO DO IN THIS PART IS 
% TO FILL IN THE HOMEWORK NUMBER, YOUR NAME AND LOG IN BELOW
% REPLACE ``X'' WITH THE HW NUMBER, ``Your Name'' WITH YOUR NAME,
% AND ``your login'' WITH YOUR LOGIN.

\newcommand{\hwnumber}{12}
\newcommand{\yourname}{Michael Mueller}

% NOW YOU MAY SKIP DOWN TO THE PART CALLED ``YOUR DOCUMENT''.

%==============================================================================
% Formatting parameters (how the page is set up)
%==============================================================================
\newcommand{\yourcourse}{Math 695}
\newcommand{\chapter}{0}
\newcommand{\mgn}{\mathcal M_{g,n}}
\newcommand{\mgnb}{\overline{\mathcal M_{g,n}}}
\newcommand{\mgb}[1]{\overline{\mathcal M_{g,#1}}}

\documentclass[11pt]{article}           % 11pt article
\makeatletter                   % Make '@' accessible.
\pagestyle{myheadings}              % We do our own page headers.
\newcommand{\thishw}{\bf Elliptic curve question}
\def\@oddhead{\bf \thishw \hfill \yourname}
\oddsidemargin=0in              % Left margin minus 1 inch.
\evensidemargin=0in             % Same for even-numbered pages.
\textwidth=6.5in                % Text width (8.5in - margins).
\topmargin=0in                  % Top margin minus 1 inch.
\headsep=0.2in                  % Distance from header to body.
\textheight=8in                 % Body height (incl. footnotes)
\skip\footins=4ex               % Space above first footnote.
\hbadness=10000                 % No "underfull hbox" messages.
\makeatother                    % Make '@' special again.

%==============================================================================
% Packages used (packages add more commands)
%==============================================================================

\usepackage{amsmath}                % give more fonts and symbols
\usepackage{amsfonts}               % want AMS fonts
\usepackage{amssymb}
\usepackage{amsthm}
\usepackage{mathrsfs}
\usepackage{tikz-cd}
\usepackage{bbm}                % given mathbbm fonts
\usepackage[shortlabels]{enumitem}
\usepackage{relsize}
\usepackage{hyperref}

\makeatletter
\newcommand*{\relrelbarsep}{.386ex}
\newcommand*{\relrelbar}{%
  \mathrel{%
    \mathpalette\@relrelbar\relrelbarsep
  }%
}
\newcommand*{\@relrelbar}[2]{%
  \raise#2\hbox to 0pt{$\m@th#1\relbar$\hss}%
  \lower#2\hbox{$\m@th#1\relbar$}%
}
\providecommand*{\rightrightarrowsfill@}{%
  \arrowfill@\relrelbar\relrelbar\rightrightarrows
}
\providecommand*{\leftleftarrowsfill@}{%
  \arrowfill@\leftleftarrows\relrelbar\relrelbar
}
\providecommand*{\xrightrightarrows}[2][]{%
  \ext@arrow 0359\rightrightarrowsfill@{#1}{#2}%
}
\providecommand*{\xleftleftarrows}[2][]{%
  \ext@arrow 3095\leftleftarrowsfill@{#1}{#2}%
}
\makeatother

%==============================================================================
% Macros (make your own commands)
%==============================================================================

% For problem and part headers
\newcounter{problemcounter}
\newcounter{subproblemcounter}
\newcommand{\problem}{
    \addtocounter{problemcounter}{1}
    \bigskip
    \noindent {\Large Problem \hwnumber .\theproblemcounter}
    \smallskip
    \setcounter{subproblemcounter}{0}
}
\newcommand{\subproblem}{
    \addtocounter{subproblemcounter}{1}
    \smallskip
    \noindent {\bf \alph{subproblemcounter})} 
}

% Nice things
\newcommand{\set}[1]{\{#1\}}            % Set (as in \set{1,2,3})
\newcommand{\setof}[2]{\{\,{#1}|~{#2}\,\}}  % Set (as in \setof{x}{x > 0})

% Some letter symbols
\newcommand{\N}{\ensuremath{\mathbb{N}}}
\newcommand{\Z}{\ensuremath{\mathbb{Z}}}
\newcommand{\R}{\ensuremath{\mathbb{R}}}
\newcommand{\hTop}{\textbf{hTop}}
\newtheorem*{Proposition}{Proposition}
\newtheorem*{Corollary}{Corollary}
\newcommand{\Tor}{\text{Tor}}
\newcommand{\Ext}{\text{Ext}}
\newcommand{\Q}{\mathbb{Q}}
\newcommand{\F}{\mathbb{F}}
\newcommand{\C}{\mathbb{C}}
\newcommand{\CP}{\mathbb{CP}}
\newcommand{\RP}{\mathbb{RP}}
\newcommand{\Spec}{\text{Spec}}
\newcommand{\Aut}{\text{Aut}}
\newcommand{\Proj}{\text{Proj}}
\newcommand{\Mor}{\text{Mor}}
\newcommand{\codim}{\text{codim}}
\newcommand{\exer}[1]{{\bf Exercise #1} \\}
\newcommand{\Hom}{\text{Hom}}
\newcommand{\coker}{\text{coker}}
\newcommand*\simplex{\includegraphics[scale=0.017]{simplex.png}}
\newcommand{\Sch}{\textbf{Sch}}
\newcommand{\Set}{\textbf{Set}}
\renewcommand{\a}{\mathfrak a}
\renewcommand{\b}{\mathfrak b}
\renewcommand{\P}{\mathbb P}
\theoremstyle{definition}
\newtheorem*{thm}{Theorem}
\newtheorem*{prob}{Problem}
\newtheorem*{dfn}{Definition}
\newtheorem*{claim}{Claim}
\theoremstyle{definition}
\newtheorem*{lem}{Lemma}
\newtheorem*{ex}{Exercise}
\newtheorem*{eg}{Example}
\newtheorem*{note}{Note}

\usetikzlibrary{matrix,positioning,quotes}

%==============================================================================
% YOUR DOCUMENT (start here)
%==============================================================================

\begin{document}
\definecolor{myblue}{RGB}{100,180,255}
\definecolor{mygreen}{RGB}{80,160,80}
\definecolor{myred}{RGB}{200,120,100}
\centerline{\LARGE\thishw}


Let
\[
S_{\mathfrak a,\mathfrak b}(\mu)=N_{\mu,(2,1,\dots)^{a_2},(2,1,\dots)^{b_2},(3,1,\dots)^{a_3},(3,1,\dots)^{b_3},\dots}^{\{2^{b_2}3^{b_3}\dots\}}
\]
Here $a_2,a_3,\dots$ describe unmarked ramification and $b_2,b_3,\dots$ describe marked ramification. For
simplicity, assume that all ramification is included and define $S_{\mathfrak a,\mathfrak b}(\mu)= 0$ if $\mu_i\leq 0$ for any $i$.

\begin{lem}
  If $S_{\mathfrak a,\mathfrak b}(\mu)$ is well-defined, $|\mathfrak a|=|\mu|+2$.
\end{lem}
\begin{proof}
  Let $\mathcal X$ be the moduli space of maps from a genus $1$ curve to $\mathbb P^1$, with ramification as required for $S_{\mathfrak a,\mathfrak b}(\mu)$, and with all relevant points of the genus $1$ curve marked
  (corresponding to $\mathfrak a$, $\mathfrak b$, and $\mu$). There is a map $\mathcal X\to\overline{\mathcal M}_{1,|\mathfrak b|+|\mu|}$ remembering the genus $1$ curve and the fixed points, which is finite by assumption, so $\dim(\mathcal X)=|\mathfrak b|+|\mu|$. There is also a map $\mathcal X\to\overline{\mathcal M}_{0,|\mathfrak a|+|\mathfrak b|+1}$ remembering the branch points of the target, which is finite by Hurwitz's theorem, and so
  \[
  |\mathfrak b|+|\mu|=\dim(\overline{\mathcal M}_{0,|\mathfrak a|+|\mathfrak b|+1})=|\mathfrak a|+|\mathfrak b|-2
  \]
  yielding the result.
\end{proof}

\begin{claim}
  For $n>0$, if $b_y>0$, then
  \begin{align*}
    S_{\mathfrak a,\mathfrak b}(\mu, n) &=H((n),(y,1,\dots),(n-y+1,1,\dots))S_{\mathfrak a,\mathfrak b-\{y\}}(\mu, n-y+1) \\
    &+\sum_{z\in\mathfrak a}H((y+z-n-2,n),(z,1,\dots),(y,1,\dots))S_{\mathfrak a-\{z\},\mathfrak b-\{y\}+\{y+z-n-2\}}(\mu)
  \end{align*}
\end{claim}

\begin{proof}
  We consider diagrams which stabilize to a genus $1$ curve containing $x_1,\dots,x_k$ and all but one
  of the other fixed points, attached to a genus $0$ tail containing $q$ (with ramification $n$) and $p$
  (with ramification $y$).

  First, note that the total number of branch points for any such map should be $|\mathfrak a|+|\mathfrak b|+1$.
  Restricting to the genus $1$ component, the moduli space of maps for that component has a finite map to $\overline{\mathcal M}_{1,|\mathfrak b|+k}$ and therefore has dimension $|\mathfrak b|+k$, so the number of branch points is $|\mathfrak b|+k+3$. One
  of these branch points is the node of the target genus $0$, so discounting this there are $|\mathfrak b|+k+2$ proper branch points.
  Therefore, the number of branch points outside the genus $1$ component is
  \[
  (|\mathfrak a|+|\mathfrak b|+1)-(|\mathfrak b|+k+2)=|\mathfrak a|-k-1=|\mathfrak a|-|(\mu,n)|=2
  \]
  by the lemma.
  
  A diagram with $x_1,\dots,x_k$ on the right side looks like:

\begin{tikzpicture}[thick,amat/.style={matrix of nodes,nodes in empty cells,
  row sep=3.2em,rounded corners,
  nodes={draw,solid,circle,minimum size=1.5cm}},
  dmat/.style={matrix of nodes,nodes in empty cells,row sep=3.2em,nodes={minimum size=1.5cm},draw=myred},
  fsnode/.style={fill=myblue},
  ssnode/.style={fill=mygreen}]

  \matrix[amat,nodes=fsnode] (mat1) {$1$\\
    $0$\\
    \vdots \\
  $0$\\};

  \matrix[dmat,left=2cm of mat1] (degrees1) {$d-\sum c_i$\\
    $c_1$\\
    \vdots \\
  $c_r$\\};

 \matrix[amat,right=7cm of mat1,nodes=ssnode] (mat2) {$0$\\
   \vdots\\
   $0$ \\
   $0$\\};
 \matrix[dmat,right=1.5cm of mat2] (degrees2) {$\mu_1$\\
   \vdots\\
   $\mu_k$ \\
 $n$\\};

 \draw  (mat1-1-1) edge["$\mu_1$"] (mat2-1-1)
 (mat1-1-1) edge["$\mu_k$"] (mat2-3-1)
 (mat1-1-1) edge["$n-\sum c_i$"] (mat2-4-1)
 (mat1-2-1) edge["$c_1$"] (mat2-4-1)
 (mat1-4-1) edge["$c_r$"] (mat2-4-1);

  % draw legs for left side
 \draw (mat1-1-1) -- +(95:1) node[anchor=south] {fixed points};

 % draw legs for right side
 \draw  (mat2-1-1) -- +(30:1) node[anchor=west] {$x_1$}
 (mat2-3-1) -- +(30:1) node[anchor=west] {$x_k$}
 (mat2-4-1) -- +(30:1) node[anchor=west] {$q$};

\end{tikzpicture}

There are two branch points outside the genus $1$ component, which must be the images of
$x_1,\dots,x_k,q$ and of $p$. This means there is no extra ramification on any genus $0$ component, which is why each top right bubble
connects only to the genus $1$ component. The bottom right bubble must have some ramification beyond what is pictured here, which is necessarily
$p$. So our diagram looks like:

  \begin{tikzpicture}[thick,amat/.style={matrix of nodes,nodes in empty cells,
  row sep=3.2em,rounded corners,
  nodes={draw,solid,circle,minimum size=1.5cm}},
  dmat/.style={matrix of nodes,nodes in empty cells,row sep=3.2em,nodes={minimum size=1.5cm},draw=myred},
  fsnode/.style={fill=myblue},
  ssnode/.style={fill=mygreen}]

  \matrix[amat,nodes=fsnode] (mat1) {$1$\\
    $0$\\
    \vdots \\
  $0$\\};

  \matrix[dmat,left=2cm of mat1] (degrees1) {$d-r$\\
    $1$\\
    \vdots \\
  $1$\\};

 \matrix[amat,right=7cm of mat1,nodes=ssnode] (mat2) {$0$\\
   \vdots\\
   $0$ \\
   $0$\\};
 \matrix[dmat,right=1.5cm of mat2] (degrees2) {$\mu_1$\\
   \vdots\\
   $\mu_k$ \\
 $n$\\};

 \draw  (mat1-1-1) edge["$\mu_1$"] (mat2-1-1)
 (mat1-1-1) edge["$\mu_k$"] (mat2-3-1)
 (mat1-1-1) edge["$n-r$"] (mat2-4-1)
 (mat1-2-1) edge["$1$"] (mat2-4-1)
 (mat1-4-1) edge["$1$"] (mat2-4-1);

  % draw legs for left side
 \draw (mat1-1-1) -- +(95:1) node[anchor=south] {fixed points};

 % draw legs for right side
 \draw  (mat2-1-1) -- +(30:1) node[anchor=west] {$x_1$}
 (mat2-3-1) -- +(30:1) node[anchor=west] {$x_k$}
 (mat2-4-1) -- +(30:1) node[anchor=west] {$q$}
 (mat2-4-1) -- +(330:1) node[anchor=west] {$p$};

  \end{tikzpicture}

  As there is no extra ramification outside the genus $1$ component,
  the dimension reduction of $r$ should match the codimension reduction
  corresponding to the removal of $p$, which is $y-1$. So $r=y-1$ and we
  have the first term in the claim.

  Now consider the case where $x_1,\dots,x_k,q$ are on the left side. In this
  case, the branch points not coming from the genus $1$ component will be
  the image of $p$ and the image of another point $p'$ (with ramification index $z\in\mathfrak a$). Consider the possibility that
  $x_1,\dots,x_k$ are not all on the genus $1$ component:

  \begin{tikzpicture}[thick,amat/.style={matrix of nodes,nodes in empty cells,
  row sep=3.2em,rounded corners,
  nodes={draw,solid,circle,minimum size=1cm}},
  dmat/.style={matrix of nodes,nodes in empty cells,row sep=3.2em,nodes={minimum size=1cm},draw=myred},
  fsnode/.style={fill=myblue},
  ssnode/.style={fill=mygreen}]

    \matrix[amat,nodes=fsnode] (mat1) {$0$\\
      $1$\\};

    \matrix[amat,right=2cm of mat1,nodes=ssnode] (mat2) {$0$\\};

      \matrix[dmat,left=1cm of mat1] (degrees1) {$\mu_1$\\
    $d-n-\mu_1$\\};

            \matrix[dmat,right=0.5cm of mat2] (degrees2) {$e$\\};
  
 \draw  (mat1-1-1) edge["$\mu_1$"] (mat2-1-1)
 (mat1-2-1) edge["$e-\mu_1$"] (mat2-1-1);

  % draw legs for left side
 \draw (mat1-1-1) -- +(120:1) node[anchor=south] {$x_1$}
 (mat1-2-1) -- +(120:1) node[anchor=south] {$x_2,\dots,x_k$};

   % draw legs for right side
 \draw (mat2-1-1) -- +(100:1) node[anchor=south] {$e$ ramif};


  \end{tikzpicture}

  This would require the genus $0$ on the right side to have multiple branch points contributing
  $e$ ramification,
  but $p$ cannot live on this component (because of the stabilization condition), a contradiction.
  Therefore, $x_1,\dots,x_k$ all live on the genus $1$ component.

  The points $p$ and $p'$ cannot lie on separate components, because the following diagram cannot be completed:

        \begin{tikzpicture}[thick,amat/.style={matrix of nodes,nodes in empty cells,
  row sep=3.2em,rounded corners,
  nodes={draw,solid,circle,minimum size=1.5cm}},
  dmat/.style={matrix of nodes,nodes in empty cells,row sep=3.2em,nodes={minimum size=1.5cm},draw=myred},
  fsnode/.style={fill=myblue},
  ssnode/.style={fill=mygreen}]

  \matrix[amat,nodes=fsnode] (mat1) {$1$\\
    $0$\\};

  \matrix[dmat,left=2cm of mat1] (degrees1) {$d-n$\\
    $n$\\};

 \matrix[amat,right=3cm of mat1,nodes=ssnode] (mat2) {$0$\\
   $0$\\};
 \matrix[dmat,right=1.5cm of mat2] (degrees2) {$z$\\
   $y$\\};

 \draw  (mat1-2-1) edge["$y$"] (mat2-2-1);

  % draw legs for left side
 \draw (mat1-1-1) -- +(115:1) node[anchor=south] {fixed points}
 (mat1-1-1) -- +(260:1) node[anchor=north] {$x_1,\dots,x_k$}
 (mat1-2-1) -- +(260:1) node[anchor=east] {$q$};

   % draw legs for right side
 \draw (mat2-1-1) -- +(75:1) node[anchor=south] {$p'$}
 (mat2-2-1) -- +(265:1) node[anchor=north] {$p$};


 \end{tikzpicture}

  Therefore $p$ and $p'$ lie on the same genus $0$ component, all other genus $0$ components on the right side have degree $1$,
  and we end up with:
  
        \begin{tikzpicture}[thick,amat/.style={matrix of nodes,nodes in empty cells,
  row sep=3.2em,rounded corners,
  nodes={draw,solid,circle,minimum size=1.5cm}},
  dmat/.style={matrix of nodes,nodes in empty cells,row sep=3.2em,nodes={minimum size=1.5cm},draw=myred},
  fsnode/.style={fill=myblue},
  ssnode/.style={fill=mygreen}]

  \matrix[amat,nodes=fsnode] (mat1) {$1$\\
    $0$\\};

  \matrix[dmat,left=2cm of mat1] (degrees1) {$d-n$\\
    $n$\\};

 \matrix[amat,right=3cm of mat1,nodes=ssnode] (mat2) {$0$\\
   \vdots\\
   $0$ \\
   $0$\\};
 \matrix[dmat,right=1.5cm of mat2] (degrees2) {$1$\\
   \vdots\\
   $1$ \\
   $e+n$\\};

 \draw  (mat1-1-1) edge["$1$"] (mat2-1-1)
 (mat1-1-1) edge["$1$"] (mat2-3-1)
 (mat1-1-1) edge["$e$"] (mat2-4-1)
 (mat1-2-1) edge["$n$"] (mat2-4-1);

  % draw legs for left side
 \draw (mat1-1-1) -- +(115:1) node[anchor=south] {fixed points}
 (mat1-1-1) -- +(260:1) node[anchor=north] {$x_1,\dots,x_k$}
 (mat1-2-1) -- +(140:1) node[anchor=east] {$q$};

   % draw legs for right side
 \draw (mat2-4-1) -- +(75:1) node[anchor=south] {$p$}
 (mat2-4-1) -- +(265:1) node[anchor=north] {$p'$};


        \end{tikzpicture}

        By Riemann-Hurwitz for the bottom-right component, 
        \[
        (e+n-2)+(y-1)+(z-1)=2(e+n)-2\implies y+z-2=e+n\implies e=y+z-n-2.
        \]
        
\end{proof}


\begin{claim}
  For $n>0$,
  \begin{align*}
    S_{\mathfrak a,\mathfrak b}(\mu, n, m) &= \sum_{z\in\mathfrak a}\hat H((n,m),(z,1,\dots),(n+m-z+2,1,\dots))S_{\a-\{z\},\b}(\mu,n+m-z+2) \\
    &+\sum_{z_1,z_2\in\a}H((z_1+z_2-n-m-3,n,m),(z_1,1,\dots),(z_2,1,\dots))S_{\a-\{z_1,z_2\},\b+\{z_1+z_2-n-m-3\}}(\mu)
  \end{align*}
\end{claim}
Here $\hat H$ means ``multiply by $2$ if $n=m$''.

\begin{proof}
    We consider diagrams which stabilize to a genus $1$ curve containing $x_1,\dots,x_k$ and all 
  of the other fixed points, attached to a genus $0$ tail containing $q$ (with ramification index $n$) and $r$
  (with ramification index $m$).
  
  By reasoning as in the last claim, the number of branch points not coming from the genus $1$ component is
  \[
  (|\mathfrak a|+|\mathfrak b|+1)-(|\b|+k+3)=|\a|-|(\mu,n,m)|=2.
  \]
  If $x_1,\dots,x_k,q,r$ are on the right side, then there is one more branch point arising as the image of some $s$ (with
  ramification index $z$). Suppose $q$ and $r$ are on different components:

  \begin{tikzpicture}[thick,amat/.style={matrix of nodes,nodes in empty cells,
  row sep=3.2em,rounded corners,
  nodes={draw,solid,circle,minimum size=1.5cm}},
  dmat/.style={matrix of nodes,nodes in empty cells,row sep=3.2em,nodes={minimum size=1.5cm},draw=myred},
  fsnode/.style={fill=myblue},
  ssnode/.style={fill=mygreen}]

  \matrix[amat,nodes=fsnode] (mat1) {$1$\\
    $0$\\};

 \matrix[amat,right=7cm of mat1,nodes=ssnode] (mat2) {$0$\\
   \vdots\\
   $0$ \\
   $0$\\
 $0$\\};

 \draw  (mat1-1-1) edge["$\mu_1$"] (mat2-1-1)
 (mat1-1-1) edge["$\mu_k$"] (mat2-3-1)
 (mat1-1-1) edge["$c$"] (mat2-4-1)
 (mat1-2-1) edge["$e$"] (mat2-4-1)
 (mat1-2-1) edge["$m$"] (mat2-5-1);

  % draw legs for left side
 \draw (mat1-1-1) -- +(95:1) node[anchor=south] {fixed points};

 % draw legs for right side
 \draw  (mat2-1-1) -- +(30:1) node[anchor=west] {$x_1$}
 (mat2-3-1) -- +(30:1) node[anchor=west] {$x_k$}
 (mat2-4-1) -- +(30:1) node[anchor=west] {$q$}
 (mat2-5-1) -- +(30:1) node[anchor=west] {$r$};

  \end{tikzpicture}

  (Without loss of generality, we suppose the component containing $q$ is the one that connects to the genus $1$.)
  There is extra ramification on both the component containing $q$ and the lower left component, a contradiction. Therefore
  $q$ and $r$ lie on the same component, and our picture must look like:

    \begin{tikzpicture}[thick,amat/.style={matrix of nodes,nodes in empty cells,
  row sep=3.2em,rounded corners,
  nodes={draw,solid,circle,minimum size=1.5cm}},
  dmat/.style={matrix of nodes,nodes in empty cells,row sep=3.2em,nodes={minimum size=1.5cm},draw=myred},
  fsnode/.style={fill=myblue},
  ssnode/.style={fill=mygreen}]

  \matrix[amat,nodes=fsnode] (mat1) {$1$\\
    $0$\\
    \vdots\\
    $0$\\};

    \matrix[dmat,left=1cm of mat1] (degrees1) {$d-e$\\
    $1$\\
    \vdots \\
  $1$\\};


 \matrix[amat,right=8cm of mat1,nodes=ssnode] (mat2) {$0$\\
   \vdots\\
   $0$ \\
   $0$\\};

     \matrix[dmat,right=1cm of mat2] (degrees2) {$\mu_1$\\
    \vdots \\
    $\mu_k$\\
     $n+m$\\};

 \draw  (mat1-1-1) edge["$\mu_1$"] (mat2-1-1)
 (mat1-1-1) edge["$\mu_k$"] (mat2-3-1)
 (mat1-1-1) edge["$n+m-e$"] (mat2-4-1)
 (mat1-2-1) edge["$1$"] (mat2-4-1)
 (mat1-4-1) edge["$1$"] (mat2-4-1);

  % draw legs for left side
 \draw (mat1-1-1) -- +(95:1) node[anchor=south] {fixed points};

 % draw legs for right side
 \draw  (mat2-1-1) -- +(30:1) node[anchor=west] {$x_1$}
 (mat2-3-1) -- +(30:1) node[anchor=west] {$x_k$}
 (mat2-4-1) -- +(30:1) node[anchor=west] {$q$}
 (mat2-4-1) -- +(50:1) node[anchor=west] {$r$}
 (mat2-4-1) -- +(300:1) node[anchor=west] {$s$};

    \end{tikzpicture}

    By Riemann-Hurwitz for the lower right component,
    \[
    (n+m-e-1)+(n+m-2)+(z-1)=2(n+m)-2\implies e=z-2
    \]
    yielding the first term in the claim.

    If $x_1,\dots,x_k,q,r$ are on the left side, then there are two branch points arising as the images of $s_1$ (with
    ramification index $z_1$) and $s_2$ (with ramification index $z_2$). By logic similar to the last proof, the
    $x_1,\dots,x_k$ all lie on the genus $1$ component.

    Consider the possibility that $q$ and $r$ lie on the same component:

      \begin{tikzpicture}[thick,amat/.style={matrix of nodes,nodes in empty cells,
  row sep=3.2em,rounded corners,
  nodes={draw,solid,circle,minimum size=1.5cm}},
  dmat/.style={matrix of nodes,nodes in empty cells,row sep=3.2em,nodes={minimum size=1.5cm},draw=myred},
  fsnode/.style={fill=myblue},
  ssnode/.style={fill=mygreen}]

  \matrix[amat,nodes=fsnode] (mat1) {$1$\\
    $0$\\};

 \matrix[amat,right=7cm of mat1,nodes=ssnode] (mat2) {$0$\\
   \vdots\\
   $0$ \\
 $0$\\};

 \draw  (mat1-1-1) edge["$1$"] (mat2-1-1)
 (mat1-1-1) edge["$1$"] (mat2-3-1)
 (mat1-1-1) edge["$c$"] (mat2-4-1)
 (mat1-2-1) edge["$n+m$"] (mat2-4-1);

  % draw legs for left side
 \draw (mat1-1-1) -- +(95:1) node[anchor=south] {fixed points}
 (mat1-1-1) -- +(270:1) node[anchor=north] {$x_1,\dots,x_k$}
 (mat1-2-1) -- +(270:1) node[anchor=north] {$q$}
 (mat1-2-1) -- +(255:1) node[anchor=north] {$r$};

      \end{tikzpicture}

      There must be at least one additional branch point from the lower left component and at least two additional branch
      points from the lower right component, a contradiction. Therefore, $q$ and $r$ lie on separate components:

          \begin{tikzpicture}[thick,amat/.style={matrix of nodes,nodes in empty cells,
  row sep=3.2em,rounded corners,
  nodes={draw,solid,circle,minimum size=1.5cm}},
  dmat/.style={matrix of nodes,nodes in empty cells,row sep=3.2em,nodes={minimum size=1.5cm},draw=myred},
  fsnode/.style={fill=myblue},
  ssnode/.style={fill=mygreen}]

  \matrix[amat,nodes=fsnode] (mat1) {$1$\\
    $0$\\
    $0$\\};

    \matrix[dmat,left=1cm of mat1] (degrees1) {$d-n-m$\\
    $n$\\
  $m$\\};


 \matrix[amat,right=4cm of mat1,nodes=ssnode] (mat2) {$0$\\
   \vdots\\
   $0$ \\
   $0$\\};

     \matrix[dmat,right=1cm of mat2] (degrees2) {$1$\\
    \vdots \\
    $1$\\
     $c+n+m$\\};

 \draw  (mat1-1-1) edge["$1$"] (mat2-1-1)
 (mat1-1-1) edge["$1$"] (mat2-3-1)
 (mat1-1-1) edge["$c$"] (mat2-4-1)
 (mat1-2-1) edge["$n$"] (mat2-4-1)
 (mat1-3-1) edge["$m$"] (mat2-4-1);

  % draw legs for left side
 \draw (mat1-1-1) -- +(95:1) node[anchor=south] {fixed points}
 (mat1-1-1) -- +(245:1) node[anchor=north] {$x_1,\dots,x_k$}
 (mat1-2-1) -- +(245:1) node[anchor=north] {$q$}
 (mat1-3-1) -- +(245:1) node[anchor=north] {$r$};

 % draw legs for right side
 \draw  (mat2-4-1) -- +(30:1) node[anchor=west] {$s_1$}
 (mat2-4-1) -- +(300:1) node[anchor=west] {$s_2$};

          \end{tikzpicture}
          
          By Riemann-Hurwitz for the lower-right component,
          \[
          (c+n+m-3)+(z_1-1)+(z_2-1)=2(c+n+m)-2\implies c=z_1+z_2-n-m-3
          \]
          yielding the second term of the claim.
  \end{proof}

\end{document}
