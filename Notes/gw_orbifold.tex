% This is a Homework Template for CS 1010.  
% It was modified by  Michael Carl Tschantz (mtschant) 
% who provided the ``useful infomation on latex''.

% THE ONLY THING YOU NEED TO DO IN THIS PART IS 
% TO FILL IN THE HOMEWORK NUMBER, YOUR NAME AND LOG IN BELOW
% REPLACE ``X'' WITH THE HW NUMBER, ``Your Name'' WITH YOUR NAME,
% AND ``your login'' WITH YOUR LOGIN.

\newcommand{\hwnumber}{12}
\newcommand{\yourname}{Michael Mueller}

% NOW YOU MAY SKIP DOWN TO THE PART CALLED ``YOUR DOCUMENT''.

%==============================================================================
% Formatting parameters (how the page is set up)
%==============================================================================
\newcommand{\yourcourse}{Math 695}
\newcommand{\chapter}{0}
\newcommand{\mgn}{\mathcal M_{g,n}}
\newcommand{\mgnb}{\overline{\mathcal M_{g,n}}}
\newcommand{\mgb}[1]{\overline{\mathcal M_{g,#1}}}

\documentclass[11pt]{article}           % 11pt article
\makeatletter                   % Make '@' accessible.
\pagestyle{myheadings}              % We do our own page headers.
\newcommand{\thishw}{\bf Elliptic curve question}
\def\@oddhead{\bf \thishw \hfill \yourname}
\oddsidemargin=0in              % Left margin minus 1 inch.
\evensidemargin=0in             % Same for even-numbered pages.
\textwidth=6.5in                % Text width (8.5in - margins).
\topmargin=0in                  % Top margin minus 1 inch.
\headsep=0.2in                  % Distance from header to body.
\textheight=8in                 % Body height (incl. footnotes)
\skip\footins=4ex               % Space above first footnote.
\hbadness=10000                 % No "underfull hbox" messages.
\makeatother                    % Make '@' special again.

%==============================================================================
% Packages used (packages add more commands)
%==============================================================================

\usepackage{amsmath}                % give more fonts and symbols
\usepackage{amsfonts}               % want AMS fonts
\usepackage{amssymb}
\usepackage{amsthm}
\usepackage{mathrsfs}
\usepackage{tikz-cd}
\usepackage{bbm}                % given mathbbm fonts
\usepackage[shortlabels]{enumitem}
\usepackage{relsize}
\usepackage{hyperref}

\makeatletter
\newcommand*{\relrelbarsep}{.386ex}
\newcommand*{\relrelbar}{%
  \mathrel{%
    \mathpalette\@relrelbar\relrelbarsep
  }%
}
\newcommand*{\@relrelbar}[2]{%
  \raise#2\hbox to 0pt{$\m@th#1\relbar$\hss}%
  \lower#2\hbox{$\m@th#1\relbar$}%
}
\providecommand*{\rightrightarrowsfill@}{%
  \arrowfill@\relrelbar\relrelbar\rightrightarrows
}
\providecommand*{\leftleftarrowsfill@}{%
  \arrowfill@\leftleftarrows\relrelbar\relrelbar
}
\providecommand*{\xrightrightarrows}[2][]{%
  \ext@arrow 0359\rightrightarrowsfill@{#1}{#2}%
}
\providecommand*{\xleftleftarrows}[2][]{%
  \ext@arrow 3095\leftleftarrowsfill@{#1}{#2}%
}
\makeatother

%==============================================================================
% Macros (make your own commands)
%==============================================================================

% For problem and part headers
\newcounter{problemcounter}
\newcounter{subproblemcounter}
\newcommand{\problem}{
    \addtocounter{problemcounter}{1}
    \bigskip
    \noindent {\Large Problem \hwnumber .\theproblemcounter}
    \smallskip
    \setcounter{subproblemcounter}{0}
}
\newcommand{\subproblem}{
    \addtocounter{subproblemcounter}{1}
    \smallskip
    \noindent {\bf \alph{subproblemcounter})} 
}

% Nice things
\newcommand{\set}[1]{\{#1\}}            % Set (as in \set{1,2,3})
\newcommand{\setof}[2]{\{\,{#1}|~{#2}\,\}}  % Set (as in \setof{x}{x > 0})

% Some letter symbols
\newcommand{\N}{\ensuremath{\mathbb{N}}}
\newcommand{\Z}{\ensuremath{\mathbb{Z}}}
\newcommand{\R}{\ensuremath{\mathbb{R}}}
\newcommand{\hTop}{\textbf{hTop}}
\newtheorem*{Proposition}{Proposition}
\newtheorem*{Corollary}{Corollary}
\newcommand{\Tor}{\text{Tor}}
\newcommand{\Ext}{\text{Ext}}
\newcommand{\Q}{\mathbb{Q}}
\newcommand{\F}{\mathbb{F}}
\newcommand{\C}{\mathbb{C}}
\newcommand{\CP}{\mathbb{CP}}
\newcommand{\RP}{\mathbb{RP}}
\newcommand{\Spec}{\text{Spec}}
\newcommand{\Aut}{\text{Aut}}
\newcommand{\Proj}{\text{Proj}}
\newcommand{\Mor}{\text{Mor}}
\newcommand{\codim}{\text{codim}}
\newcommand{\exer}[1]{{\bf Exercise #1} \\}
\newcommand{\Hom}{\text{Hom}}
\newcommand{\coker}{\text{coker}}
\newcommand*\simplex{\includegraphics[scale=0.017]{simplex.png}}
\newcommand{\Sch}{\textbf{Sch}}
\newcommand{\Set}{\textbf{Set}}
\renewcommand{\P}{\mathbb P}
\theoremstyle{definition}
\newtheorem*{thm}{Theorem}
\newtheorem*{prob}{Problem}
\newtheorem*{dfn}{Definition}
\newtheorem*{claim}{Claim}
\theoremstyle{definition}
\newtheorem*{lem}{Lemma}
\newtheorem*{ex}{Exercise}
\newtheorem*{eg}{Example}
\newtheorem*{note}{Note}

%==============================================================================
% YOUR DOCUMENT (start here)
%==============================================================================

\begin{document}
\centerline{\LARGE\thishw}
\begin{prob}
  Consider the relative moduli stack $\overline{\mathcal M}_{1,1}(\P^1_{\Z/2,\Z/2},(d))$ of degree $d$ stable maps from an elliptic curve $(E,p)$
  to the orbifold $\P^1_{\Z/2,\Z/2}$ ($\P^1$ with $\Z/2$ stabilizer at $1$ and $\infty$) fully ramified at $p$ over $0$.
  There is a map $\pi$ from this moduli stack to $\overline{\mathcal M}_{1,1}$, and the image of the virtual fundamental class is an element of $A^0(\overline{\mathcal M}_{1,1})\cong \Z$.
  What is this number $N_d$?
\end{prob}

Alternatively:

\begin{prob}
  Let $(E,p)$ be an elliptic curve. How many maps $f:(E,p)\to (\P^1,0)$ are there such that $f$ has ramification profile $(d)$ over $0$, $(2)^d$ over $1$, and $(2)^d$ over $\infty$, weighted by automorphisms? (Let's call such a map {\bf special}.)
\end{prob}

First nontrivial case I studied is $d=4$:
See ``Documents/four\_curves\_linear.m2'' for the computation in M2. The idea is to consider
\[
\P^3\supset A=\{(x,y)\in E^2:x+y=\text{2-torsion}\}\to B=\{H:H\cap E=2x+2y\}\subset(\P^3)^*
\]
Both of these have 4 connected components: $A_i\cong E$ and $B_i\cong\P^1$, quotienting by the elliptic involution. Then $B$ has a map
to $\P^2$ (locus of lines contained in $H_0$) sending $H\mapsto H\cap H_0$, with image $C$. See my notes for a picture of $C$;
there are 4 singular points of $C$, corresponding to $3$ valid lines (and one where the map $E\to\ell$ is undefined). For each other line,
we have a map $E\to \ell$ and can postcompose by a map $\P^1\cong\P^1$ fixing $0$ and interchanging $1$ and $\infty$, so 2 maps per line.
Each map has 2 automorphisms, so $N_4=3\cdot 2\cdot\frac 12=3$.

For the general case, I claim that $N_d=3$ for $d$ even and $N_d=0$ for $d$ odd. The idea here is that every special map factors through the degree 2 map $E\to\P^1$. To see this, we can do a Hurwitz count. TODO...

\includegraphics[scale=0.2]{diagram.jpg}

\includegraphics[angle=90,scale=0.15]{diagram2.jpg}

\begin{claim}
  Let $f:E\to\P^1$ be of degree $d=2e$ such that $f(p)=0$ with full ramification,
  and $f$ has ramification profiles $(2)^e$ over $1$ and $\infty$. Then $f$ factors as
  $E\to\P^1\to\P^1$ where the first map is degree $2$ and the second is degree $e$.
\end{claim}
\begin{proof}
  Let $w\in\P^1$ be the point over which $f$ has ramification profile $(2,1,1,\dots,1)$, which exists by Riemann-Hurwitz. I will
  count the number of Hurwitz covers over $0,1,\infty,w$ which factor (i.e., count the degree $e$ maps pictured above) and the
  total number of Hurwitz covers (in the 2nd picture), and show these are the same. The first
  count is $e$, since there is one unique map $\P^1\to\P^1$ by the count in the proof of the claim below, and there are then $e$ choices
  for $d$ (the 4th ramification point of $E\to\P^1$) out of all the preimages of $w$.

  For the second count...TODO
\end{proof}
\begin{claim}
  The count of maps in the last claim is $6$. Each such map has 2 automorphisms, so the weighted count is $\frac 62=3$.
\end{claim}
\begin{proof}

  The first map $E\to\P^1$ is unique up to choice of 4-tuple $(a,b,c,d)$, so we want to count maps $\P^1\to\P^1$ of the required
  form. To form a map as pictured, we have $3$ choices for which of $b,c,d$ should lie
  over an extra point $w$, and $2$ choices for whether a
  fixed one of the remaining $2$ points lies over $1$ or $\infty$,
  so $6$ choices in total. It remains to show that there is a unique map pictured
  once $a,b,c,d$ have been worked out, which can be shown by a Hurwitz count.
  
  In the case where $e$ is odd, we want to count the number of triples $(\sigma_1,\sigma_2,\sigma_3)$
  such that $\sigma_1\sigma_2=\sigma_3$ and $\sigma_1,\sigma_2$ are a product of
  $(e-1)/2$ disjoint transpositions while $\sigma_3$ is an $e$-cycle.
  Let's assume
  \[
  \sigma_1=(1\ 2)(3\ 4)\cdots (e-2\ e-1).
  \]
  Then $\sigma_2$ can't fix $e$ (otherwise $\sigma_3$ would), so $\sigma_2$
  has a transposition of the form $(e\ x_1)$. The following claim will be used to show that up to relabeling, the only way to do this is
  \[
  (1\ 2)(3\ 4)\cdots (e-2\ e-1)(e\ 1)(2\ 3)(4\ 5)\cdots (e-3\ e-2)=(e\ 2\ 4\ 6\ \cdots\ e-1\ e-2\ e-4\ \dots\ 1)
  \]

  {\bf Claim:} Suppose $\sigma_1\sigma_2$ is an $e$-cycle. For all odd $i\leq e-2$,
  $\sigma_2$ includes disjoint transpositions $(x_0\ x_1)(x_2\ x_3)\cdots (x_{i-1}\ x_i)$
  where $x_0=e$ and $\sigma_1$ has transpositions $(x_1\ x_2)\cdots (x_{i-2}\ x_{i-1})$.

  {\bf Proof:} The base case $i=1$ holds since $\sigma_2$ includes the transposition $(e\ x_1)$.
  Now suppose it holds for $i$,
  where $i<e-2$; I claim that it holds for $i+2$ as well. Let $\sigma_1$ have the transposition $(x_i\ x_{i+1})$ (this exists
  since $x_i\neq e$).
  Note that $\sigma_1\sigma_2$ sends
  $x_j\mapsto x_{j+2}$ for even $j$ where $j\leq i-1$. If $\sigma_2$ fixes $x_{i+1}$, then $\sigma_1\sigma_2$ will include the cycle
  \[
  x_0\mapsto x_2\mapsto x_4\mapsto\cdots\mapsto x_{i+1}\mapsto x_i\mapsto x_{i-2}\mapsto\cdots\mapsto x_1\mapsto x_0
  \]
  This is a cycle of length $i+2<e$, contradicting the assumption that $\sigma_1\sigma_2$ is an $e$-cycle. Therefore
  $\sigma_1$ doesn't fix $x_{i+1}$ and so it includes a transposition $(x_{i+1}\ x_{i+2})$, as claimed.

  Letting $i=e-2$, this shows that
  \[
  \sigma_2=(x_0\ x_1)(x_2\ x_3)\cdots (x_{e-3}\ x_{e-2}).
  \]
  (TODO: show that $\sigma_2$ of this form do actually work.) There are $e-1$ options for $x_1$, $e-3$ options for $x_3$, and so on,
  for a total of $(e-1)(e-3)\cdots (4)(2)=2^{(e-1)/2}((e-1)/2)!$ choices.
  There are
  \[
  \binom{e}{2}\cdot\binom{e-2}{2}\cdots\binom{3}{2}\cdot\frac 1{((e-1)/2)!}
  \]
  choices for $\sigma_1$, so multiplying these together we get a total of
  \[
  2^{(e-1)/2}\cdot\frac{e(e-1)(e-2)(e-3)\cdots (3)(2)}{2^{(e-1)/2}}=e!
  \]
  choices. Therefore the Hurwitz count is $1$, and since these covers have no automorphisms (as $a,b,c$ are fixed), it is an enumerative count.
  A similar technique will work for $e$ even.

  \end{proof}


\end{document}
