% This is a Homework Template for CS 1010.  
% It was modified by  Michael Carl Tschantz (mtschant) 
% who provided the ``useful infomation on latex''.

% THE ONLY THING YOU NEED TO DO IN THIS PART IS 
% TO FILL IN THE HOMEWORK NUMBER, YOUR NAME AND LOG IN BELOW
% REPLACE ``X'' WITH THE HW NUMBER, ``Your Name'' WITH YOUR NAME,
% AND ``your login'' WITH YOUR LOGIN.

\newcommand{\hwnumber}{12}
\newcommand{\yourname}{Michael Mueller}

% NOW YOU MAY SKIP DOWN TO THE PART CALLED ``YOUR DOCUMENT''.

%==============================================================================
% Formatting parameters (how the page is set up)
%==============================================================================
\newcommand{\yourcourse}{Math 695}
\newcommand{\chapter}{0}
\newcommand{\mgn}{\mathcal M_{g,n}}
\newcommand{\mgnb}{\overline{\mathcal M_{g,n}}}
\newcommand{\mgb}[1]{\overline{\mathcal M_{g,#1}}}

\documentclass[11pt]{article}           % 11pt article
\makeatletter                   % Make '@' accessible.
\pagestyle{myheadings}              % We do our own page headers.
\newcommand{\thishw}{\bf Elliptic curve question}
\def\@oddhead{\bf \thishw \hfill \yourname}
\oddsidemargin=0in              % Left margin minus 1 inch.
\evensidemargin=0in             % Same for even-numbered pages.
\textwidth=6.5in                % Text width (8.5in - margins).
\topmargin=0in                  % Top margin minus 1 inch.
\headsep=0.2in                  % Distance from header to body.
\textheight=8in                 % Body height (incl. footnotes)
\skip\footins=4ex               % Space above first footnote.
\hbadness=10000                 % No "underfull hbox" messages.
\makeatother                    % Make '@' special again.

%==============================================================================
% Packages used (packages add more commands)
%==============================================================================

\usepackage{amsmath}                % give more fonts and symbols
\usepackage{amsfonts}               % want AMS fonts
\usepackage{amssymb}
\usepackage{amsthm}
\usepackage{mathrsfs}
\usepackage{tikz-cd}
\usepackage{bbm}                % given mathbbm fonts
\usepackage[shortlabels]{enumitem}
\usepackage{relsize}
\usepackage{hyperref}

\makeatletter
\newcommand*{\relrelbarsep}{.386ex}
\newcommand*{\relrelbar}{%
  \mathrel{%
    \mathpalette\@relrelbar\relrelbarsep
  }%
}
\newcommand*{\@relrelbar}[2]{%
  \raise#2\hbox to 0pt{$\m@th#1\relbar$\hss}%
  \lower#2\hbox{$\m@th#1\relbar$}%
}
\providecommand*{\rightrightarrowsfill@}{%
  \arrowfill@\relrelbar\relrelbar\rightrightarrows
}
\providecommand*{\leftleftarrowsfill@}{%
  \arrowfill@\leftleftarrows\relrelbar\relrelbar
}
\providecommand*{\xrightrightarrows}[2][]{%
  \ext@arrow 0359\rightrightarrowsfill@{#1}{#2}%
}
\providecommand*{\xleftleftarrows}[2][]{%
  \ext@arrow 3095\leftleftarrowsfill@{#1}{#2}%
}
\makeatother

%==============================================================================
% Macros (make your own commands)
%==============================================================================

% For problem and part headers
\newcounter{problemcounter}
\newcounter{subproblemcounter}
\newcommand{\problem}{
    \addtocounter{problemcounter}{1}
    \bigskip
    \noindent {\Large Problem \hwnumber .\theproblemcounter}
    \smallskip
    \setcounter{subproblemcounter}{0}
}
\newcommand{\subproblem}{
    \addtocounter{subproblemcounter}{1}
    \smallskip
    \noindent {\bf \alph{subproblemcounter})} 
}

% Nice things
\newcommand{\set}[1]{\{#1\}}            % Set (as in \set{1,2,3})
\newcommand{\setof}[2]{\{\,{#1}|~{#2}\,\}}  % Set (as in \setof{x}{x > 0})

% Some letter symbols
\newcommand{\N}{\ensuremath{\mathbb{N}}}
\newcommand{\Z}{\ensuremath{\mathbb{Z}}}
\newcommand{\R}{\ensuremath{\mathbb{R}}}
\newcommand{\hTop}{\textbf{hTop}}
\newtheorem*{Proposition}{Proposition}
\newtheorem*{Corollary}{Corollary}
\newcommand{\Tor}{\text{Tor}}
\newcommand{\Ext}{\text{Ext}}
\newcommand{\Q}{\mathbb{Q}}
\newcommand{\F}{\mathbb{F}}
\newcommand{\C}{\mathbb{C}}
\newcommand{\CP}{\mathbb{CP}}
\newcommand{\RP}{\mathbb{RP}}
\newcommand{\Spec}{\text{Spec}}
\newcommand{\Aut}{\text{Aut}}
\newcommand{\Proj}{\text{Proj}}
\newcommand{\Mor}{\text{Mor}}
\newcommand{\codim}{\text{codim}}
\newcommand{\exer}[1]{{\bf Exercise #1} \\}
\newcommand{\Hom}{\text{Hom}}
\newcommand{\coker}{\text{coker}}
\newcommand*\simplex{\includegraphics[scale=0.017]{simplex.png}}
\newcommand{\Sch}{\textbf{Sch}}
\newcommand{\Set}{\textbf{Set}}
\renewcommand{\P}{\mathbb P}
\theoremstyle{definition}
\newtheorem*{thm}{Theorem}
\newtheorem*{prob}{Problem}
\newtheorem*{dfn}{Definition}
\newtheorem*{claim}{Claim}
\theoremstyle{definition}
\newtheorem*{lem}{Lemma}
\newtheorem*{ex}{Exercise}
\newtheorem*{eg}{Example}
\newtheorem*{note}{Note}

\usetikzlibrary{matrix,positioning,quotes}

%==============================================================================
% YOUR DOCUMENT (start here)
%==============================================================================

\begin{document}
\definecolor{myblue}{RGB}{100,180,255}
\definecolor{mygreen}{RGB}{80,160,80}
\definecolor{myred}{RGB}{200,120,100}
\centerline{\LARGE\thishw}

\begin{claim}Let $S(\mu)=N(\mu,(3,1^{d-3})^{d-2})$. Then $S(\mu,1,1)=S(\mu)+S(\mu,2)$.
\end{claim}

\begin{proof}
  Let $\mu=(\mu_1,\dots,\mu_k)$. Consider an element of $\overline{\mathcal M_{1,k+2}}$
  consisting of a genus 1 curve with generic points $x_1,\dots,x_k$, attached to a
  rational tail containing points $q$ and $r$. Our main claim is that admissible
  covers from this stable curve to $\P^1$ with the desired ramification are of one
  of two forms (``Type 1'' and ``Type 2'' respectively):

\begin{tikzpicture}[thick,amat/.style={matrix of nodes,nodes in empty cells,
  row sep=2em,rounded corners,
  nodes={draw,solid,circle,minimum size=1.5cm}},
  dmat/.style={matrix of nodes,nodes in empty cells,row sep=2em,nodes={minimum size=1.5cm},draw=myred},
  fsnode/.style={fill=myblue},
  ssnode/.style={fill=mygreen}]

  \matrix[amat,nodes=fsnode] (mat1) {1\\};

   \matrix[dmat,left=2cm of mat1] (degrees1) {$d$\\};

 \matrix[amat,right=2cm of mat1,nodes=ssnode] (mat2) {0\\
   \vdots\\
   0 \\
   0\\};
 \matrix[dmat,right=1.5cm of mat2] (degrees2) {$\mu_1$\\
   \vdots\\
   $\mu_k$ \\
 $2$\\};

 \draw  (mat1-1-1) edge["$\mu_1$"] (mat2-1-1)
 (mat1-1-1) edge["$\mu_k$"] (mat2-3-1)
 (mat1-1-1) edge["$2$"] (mat2-4-1);

  % draw legs for left side
 \draw  (mat1-1-1) -- +(130:1) node[anchor=east] {$(3,1^{d-3})$}
 (mat1-1-1) -- +(200:1) node[anchor=east] {$\dots$}
 (mat1-1-1) -- +(290:1) node[anchor=north] {$(3,1^{d-3})$};

 % draw legs for right side
 \draw  (mat2-1-1) -- +(30:1) node[anchor=west] {$x_1$}
 (mat2-3-1) -- +(30:1) node[anchor=west] {$x_k$}
 (mat2-4-1) -- +(30:1) node[anchor=west] {$q$}
 (mat2-4-1) -- +(-30:1) node[anchor=west] {$r$};

\end{tikzpicture}

\begin{tikzpicture}[thick,amat/.style={matrix of nodes,nodes in empty cells,
  row sep=2.2em,rounded corners,
  nodes={draw,solid,circle,minimum size=1.5cm}},
  dmat/.style={matrix of nodes,nodes in empty cells,row sep=2.2em,nodes={minimum size=1.5cm},draw=myred},
  fsnode/.style={fill=myblue},
  ssnode/.style={fill=mygreen}]

  \matrix[amat,nodes=fsnode] (mat1) {$1$\\
    $0$\\
  $0$\\};

  \matrix[dmat,left=2cm of mat1] (degrees1) {$d-2$\\
    $1$\\
  $1$\\};

 \matrix[amat,right=2cm of mat1,nodes=ssnode] (mat2) {$0$\\
   \vdots\\
   $0$ \\
   $0$\\};
 \matrix[dmat,right=1.5cm of mat2] (degrees2) {$1$\\
   \vdots\\
   $1$ \\
 $3$\\};

 \draw  (mat1-1-1) edge["$1$"] (mat2-1-1)
 (mat1-1-1) edge["$1$"] (mat2-3-1)
 (mat1-1-1) edge["$1$"] (mat2-4-1)
 (mat1-2-1) edge["$1$"] (mat2-4-1)
 (mat1-3-1) edge["$1$"] (mat2-4-1);

  % draw legs for left side
 \draw  (mat1-1-1) -- +(130:1) node[anchor=east] {$(3,1^{d-5})$}
 (mat1-1-1) -- +(160:1) node[anchor=east] {$\dots$}
 (mat1-1-1) -- +(190:1) node[anchor=north east] {$(3,1^{d-5})$}
 (mat1-1-1) -- +(95:1) node[anchor=south] {$(x_1,\dots,x_k)$}
 (mat1-2-1) -- +(95:1) node[anchor=south] {$q$}
 (mat1-3-1) -- +(95:1) node[anchor=south] {$r$};

 % draw legs for right side
 \draw  (mat2-1-1) -- +(30:1) node[anchor=west] {$1$}
 (mat2-3-1) -- +(30:1) node[anchor=west] {$1$}
 (mat2-4-1) -- +(30:1) node[anchor=west] {$3$}
 (mat2-4-1) -- +(-30:1) node[anchor=west] {$3$};

\end{tikzpicture}

First, consider the case where $x_1,\dots,x_k,q,r$ are on the right side. Each $x_i$
must be on its own component, in order to appear on the genus 1 component after stabilization.
Also by considering stabilization, each such component must have a single edge which connects
to the genus $1$ component. For this reason, the component(s) containing $q$ and $r$ must
connect directly to the genus $1$ component, and so by stability, the only possible picture
is ``Type 1''.

Next, consider the case where $x_1,\dots,x_k,q,r$ are on the left side.
Suppose that $x_k$ does not lie on the genus $1$ component (and without loss of generality,
$x_1,\dots,x_{k-1}$ do lie on the genus $1$ component). Then we will have the following
picture:

\begin{tikzpicture}[thick,amat/.style={matrix of nodes,nodes in empty cells,
  row sep=2.2em,rounded corners,
  nodes={draw,solid,circle,minimum size=1.5cm}},
  dmat/.style={matrix of nodes,nodes in empty cells,row sep=2.2em,nodes={minimum size=1.5cm},draw=myred},
  fsnode/.style={fill=myblue},
  ssnode/.style={fill=mygreen}]

  \matrix[amat,nodes=fsnode] (mat1) {$0$\\
    $1$\\
    $0$\\
  $0$\\};

  \matrix[dmat,left=2cm of mat1] (degrees1) {$\mu_k$\\
    $d-\mu_k-2$\\
    $1$\\
  $1$\\};

  \matrix[amat,right=2cm of mat1,nodes=ssnode] (mat2) {$0$\\
    $0$ \\
   \vdots\\
   $0$ \\
   $0$\\};
  \matrix[dmat,right=1.5cm of mat2] (degrees2) {$\mu_k+a_0$\\
    $a_1$ \\
   \vdots\\
   $a_{r-1}$ \\
 $a_r+2$\\};

 \draw  (mat1-1-1) edge["$\mu_k$"] (mat2-1-1)
 (mat1-2-1) edge["$a_0$"] (mat2-1-1)
 (mat1-2-1) edge["$a_1$"] (mat2-2-1)
 (mat1-2-1) edge["$a_{r-1}$"] (mat2-4-1)
 (mat1-2-1) edge["$a_r$"] (mat2-5-1)
 (mat1-3-1) edge["$1$"] (mat2-5-1)
 (mat1-4-1) edge["$1$"] (mat2-5-1);

  % draw legs for left side
 \draw (mat1-1-1) -- +(95:1) node[anchor=south] {$x_k$}
 (mat1-2-1) -- +(130:1) node[anchor=east] {$(3,1^{d-\mu_k-5})$}
 (mat1-2-1) -- +(160:1) node[anchor=east] {$\dots$}
 (mat1-2-1) -- +(190:1) node[anchor=north east] {$(3,1^{d-\mu_k-5})$}
 (mat1-2-1) -- +(95:1) node[anchor=south] {$(x_1,\dots,x_{k-1})$}
 (mat1-3-1) -- +(95:1) node[anchor=south] {$q$}
 (mat1-4-1) -- +(95:1) node[anchor=south] {$r$};

\end{tikzpicture}

Let $s$ be the number of appearances of $(3,1^{d-\mu_k-5})$ on the genus 1 component. In order
for the invariant $N_{(x_1,\dots,x_{k-1}),(a_0,\dots,a_r),(3,1^{d-\mu_k-5})^s}^{\{a_0,a_r\}}$ to be nonzero and finite, we must have
\[
s=r-2
\]
This is because in $(\P^{d-\mu_k-3})^*$, the dimension of $X_{(a_0,\dots,a_r)}^{\{a_0,a_r\}}$ is
$(r+1)-1-2=r-2$, and in the image of $(\P^{d-\mu_k-3})^*\dashrightarrow \P^{d-\mu_k-4}$, the number
of copies of $X_{(3,1,\dots,1)}$ (codim 1) required is equal to this dimension.

Now applying Riemann-Hurwitz to the right side of the graph: the genus is $1-(r+1)=-r$, and
so
\[
2(-r)-2=d(-2)+R\implies R=2d-2r-2=2(d-r-1)
\]
This means that the number of appearances of ramification index $3$ on the right side is at
most $d-r-1$, so the number of appearances of ramification index $3$ in total is at most
\[
d-r-1+s=d-r-1+(r-2)=d-3<d-2,
\]
an impossibility. Therefore, it must be the case that $x_1,\dots,x_k$ all lie on the genus
$1$ component, yielding a picture like ``Type 2'' (but with as yet undetermined data on the right side). By another Riemann-Hurwitz calculation, the only way to fit enough ramification indices of $3$
on the right side is for all edges to be weight $1$, confirming that we must be in the ``Type 2'' scenario.
  
\end{proof}

\begin{claim} If $a+b>2$, then $S(\mu,a,b)=S(\mu,a+b)+kS(\mu,a,b)$.
\end{claim}

\begin{proof}
  First, consider the case where $x_1,\dots,x_k,q,r$ are on the right side.
  We have the following picture:

\begin{tikzpicture}[thick,amat/.style={matrix of nodes,nodes in empty cells,
  row sep=2em,rounded corners,
  nodes={draw,solid,circle,minimum size=1.5cm}},
  dmat/.style={matrix of nodes,nodes in empty cells,row sep=2em,nodes={minimum size=1.5cm},draw=myred},
  fsnode/.style={fill=myblue},
  ssnode/.style={fill=mygreen}]

  \matrix[amat,nodes=fsnode] (mat1) {$1$\\
    $0$\\
    \vdots\\
  $0$\\};

  \matrix[dmat,left=2cm of mat1] (degrees1) {$d-e$\\
    \\
    $e$\\
  \\};

 \matrix[amat,right=2cm of mat1,nodes=ssnode] (mat2) {0\\
   \vdots\\
   0 \\
   0\\
   \vdots\\
 0\\};
 \matrix[dmat,right=1.5cm of mat2] (degrees2) {$\mu_1$\\
   \vdots\\
   $\mu_k$ \\
   \\
   $a+b$\\
 \\};

 \draw  (mat1-1-1) edge["$\mu_1$"] (mat2-1-1)
 (mat1-1-1) edge["$\mu_k$"] (mat2-3-1);

  % draw legs for left side
 \draw  (mat1-1-1) -- +(130:1) node[anchor=east] {$(3,1^{d-e-3})$}
 (mat1-1-1) -- +(180:1) node[anchor=east] {$\dots$}
 (mat1-1-1) -- +(220:1) node[anchor=north] {$(3,1^{d-e-3})$};

 % draw legs for right side
 \draw  (mat2-1-1) -- +(30:1) node[anchor=west] {$x_1$}
 (mat2-3-1) -- +(30:1) node[anchor=west] {$x_k$};
\end{tikzpicture}

Let $n$ be the number of genus $0$ components on the lower right,
which must be either $1$ (containing both $q$ and $r$) or $2$ (containing
$q$ and $r$ separately). The genus of this corner is $1-n$,
and Riemann-Hurwitz gives
\[
2(1-n)=(a+b)(-2)+R\implies R=2(a+b-n)
\]

Each component has at most $e+1$ edges, leading to at least
$a+b-(e+1)$ ramification if $n=1$ or
$a-(e+1)+b-(e+1)$ ramification
if $n=2$. Considering the ramification of $q$
and $r$ as well, if there are $e$ instances of
 ramification
index $3$ on the right side, then
\[
2e+(a-1)+(b-1)+a+b-n(e+1)\leq 2a+2b-2n
\]
This simplifies to $e\leq 1$, so our picture either
looks like ``Type 1'' from the previous proof ($e=0$) or like this:

\begin{tikzpicture}[thick,amat/.style={matrix of nodes,nodes in empty cells,
  row sep=2em,rounded corners,
  nodes={draw,solid,circle,minimum size=1.5cm}},
  dmat/.style={matrix of nodes,nodes in empty cells,row sep=2em,nodes={minimum size=1.5cm},draw=myred},
  fsnode/.style={fill=myblue},
  ssnode/.style={fill=mygreen}]

  \matrix[amat,nodes=fsnode] (mat1) {$1$\\
  $0$\\};

  \matrix[dmat,left=2cm of mat1] (degrees1) {$d-1$\\
  $1$\\};

 \matrix[amat,right=2cm of mat1,nodes=ssnode] (mat2) {0\\
   \vdots\\
   0 \\
   0\\};
 \matrix[dmat,right=1.5cm of mat2] (degrees2) {$\mu_1$\\
   \vdots\\
   $\mu_k$ \\
 $a+b$\\};

 \draw  (mat1-1-1) edge["$\mu_1$"] (mat2-1-1)
 (mat1-1-1) edge["$\mu_k$"] (mat2-3-1)
 (mat1-1-1) edge["$a+b-1$"] (mat2-4-1)
 (mat1-2-1) edge["$1$"] (mat2-4-1);

  % draw legs for left side
 \draw  (mat1-1-1) -- +(130:1) node[anchor=east] {$(3,1^{d-4})$}
 (mat1-1-1) -- +(200:1) node[anchor=east] {$\dots$}
 (mat1-1-1) -- +(240:1) node[anchor=north] {$(3,1^{d-4})$};

 % draw legs for right side
 \draw  (mat2-1-1) -- +(30:1) node[anchor=west] {$x_1$}
 (mat2-3-1) -- +(30:1) node[anchor=west] {$x_k$}
 (mat2-4-1) -- +(30:1) node[anchor=west] {$q$}
 (mat2-4-1) -- +(-30:1) node[anchor=west] {$r$}
 (mat2-4-1) -- +(-100:1) node[anchor=north] {$(3,1^{a+b-3})$};

\end{tikzpicture}

The multiplicity of this factor is $k=H((a,b),(a+b-1,1),(3,1^{a+b-3}))$ (TODO: calculate).
By Riemann-Hurwitz logic similar to the last claim/proof, if $a+b>2$ then no picture with $x_1,\dots,x_k,q,r$ on the left side is possible, so we are done.
\end{proof}

We find that:
\[
S(\mu,a,1)=S(\mu,a+1)+S(\mu,a)\text{ for $a>1$}
\]
\[
S(\mu,a,2)=S(\mu,a+2)+2S(\mu,a+1)\text{ for $a\geq 2$}
\]
\[
S(\mu,3,3)=S(\mu,6)+2S(\mu,5)
\]
and $S(\mu,a,b)=0$ if $a+b>6$ (for dimension reasons; TODO how does this fit with the last claim?).

\begin{claim}
  For $d\geq 3$,
  \[
  S(2,1^{d-2})=\frac{8d^3-42d^2+82d-57}{3}
  \]
  \[
  S(1^d)=\frac{d(d-1)(2(d-1)^2+1)}{6}
  \]
  \end{claim}

\end{document}
