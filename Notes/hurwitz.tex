% This is a Homework Template for CS 1010.  
% It was modified by  Michael Carl Tschantz (mtschant) 
% who provided the ``useful infomation on latex''.

% THE ONLY THING YOU NEED TO DO IN THIS PART IS 
% TO FILL IN THE HOMEWORK NUMBER, YOUR NAME AND LOG IN BELOW
% REPLACE ``X'' WITH THE HW NUMBER, ``Your Name'' WITH YOUR NAME,
% AND ``your login'' WITH YOUR LOGIN.

\newcommand{\hwnumber}{12}
\newcommand{\yourname}{Michael Mueller}

% NOW YOU MAY SKIP DOWN TO THE PART CALLED ``YOUR DOCUMENT''.

%==============================================================================
% Formatting parameters (how the page is set up)
%==============================================================================
\newcommand{\yourcourse}{Math 695}
\newcommand{\chapter}{0}
\newcommand{\mgn}{\mathcal M_{g,n}}
\newcommand{\mgnb}{\overline{\mathcal M_{g,n}}}
\newcommand{\mgb}[1]{\overline{\mathcal M_{g,#1}}}

\documentclass[11pt]{article}           % 11pt article
\makeatletter                   % Make '@' accessible.
\pagestyle{myheadings}              % We do our own page headers.
\newcommand{\thishw}{\bf Hurwitz numbers}
\def\@oddhead{\bf \thishw \hfill \yourname}
\oddsidemargin=0in              % Left margin minus 1 inch.
\evensidemargin=0in             % Same for even-numbered pages.
\textwidth=6.5in                % Text width (8.5in - margins).
\topmargin=0in                  % Top margin minus 1 inch.
\headsep=0.2in                  % Distance from header to body.
\textheight=8in                 % Body height (incl. footnotes)
\skip\footins=4ex               % Space above first footnote.
\hbadness=10000                 % No "underfull hbox" messages.
\makeatother                    % Make '@' special again.

%==============================================================================
% Packages used (packages add more commands)
%==============================================================================

\usepackage{amsmath}                % give more fonts and symbols
\usepackage{amsfonts}               % want AMS fonts
\usepackage{amssymb}
\usepackage{amsthm}
\usepackage{mathrsfs}
\usepackage{tikz-cd}
\usepackage{bbm}                % given mathbbm fonts
\usepackage[shortlabels]{enumitem}
\usepackage{relsize}
\usepackage{hyperref}

\makeatletter
\newcommand*{\relrelbarsep}{.386ex}
\newcommand*{\relrelbar}{%
  \mathrel{%
    \mathpalette\@relrelbar\relrelbarsep
  }%
}
\newcommand*{\@relrelbar}[2]{%
  \raise#2\hbox to 0pt{$\m@th#1\relbar$\hss}%
  \lower#2\hbox{$\m@th#1\relbar$}%
}
\providecommand*{\rightrightarrowsfill@}{%
  \arrowfill@\relrelbar\relrelbar\rightrightarrows
}
\providecommand*{\leftleftarrowsfill@}{%
  \arrowfill@\leftleftarrows\relrelbar\relrelbar
}
\providecommand*{\xrightrightarrows}[2][]{%
  \ext@arrow 0359\rightrightarrowsfill@{#1}{#2}%
}
\providecommand*{\xleftleftarrows}[2][]{%
  \ext@arrow 3095\leftleftarrowsfill@{#1}{#2}%
}
\makeatother

%==============================================================================
% Macros (make your own commands)
%==============================================================================

% For problem and part headers
\newcounter{problemcounter}
\newcounter{subproblemcounter}
\newcommand{\problem}{
    \addtocounter{problemcounter}{1}
    \bigskip
    \noindent {\Large Problem \hwnumber .\theproblemcounter}
    \smallskip
    \setcounter{subproblemcounter}{0}
}
\newcommand{\subproblem}{
    \addtocounter{subproblemcounter}{1}
    \smallskip
    \noindent {\bf \alph{subproblemcounter})} 
}

% Nice things
\newcommand{\set}[1]{\{#1\}}            % Set (as in \set{1,2,3})
\newcommand{\setof}[2]{\{\,{#1}|~{#2}\,\}}  % Set (as in \setof{x}{x > 0})

% Some letter symbols
\newcommand{\N}{\ensuremath{\mathbb{N}}}
\newcommand{\Z}{\ensuremath{\mathbb{Z}}}
\newcommand{\R}{\ensuremath{\mathbb{R}}}
\newcommand{\hTop}{\textbf{hTop}}
\newtheorem*{Proposition}{Proposition}
\newtheorem*{Corollary}{Corollary}
\newcommand{\Tor}{\text{Tor}}
\newcommand{\Ext}{\text{Ext}}
\newcommand{\Q}{\mathbb{Q}}
\newcommand{\F}{\mathbb{F}}
\newcommand{\C}{\mathbb{C}}
\newcommand{\CP}{\mathbb{CP}}
\newcommand{\RP}{\mathbb{RP}}
\newcommand{\Spec}{\text{Spec}}
\newcommand{\Aut}{\text{Aut}}
\newcommand{\Proj}{\text{Proj}}
\newcommand{\Mor}{\text{Mor}}
\newcommand{\codim}{\text{codim}}
\newcommand{\exer}[1]{{\bf Exercise #1} \\}
\newcommand{\Hom}{\text{Hom}}
\newcommand{\coker}{\text{coker}}
\newcommand*\simplex{\includegraphics[scale=0.017]{simplex.png}}
\newcommand{\Sch}{\textbf{Sch}}
\newcommand{\Set}{\textbf{Set}}
\renewcommand{\a}{\mathfrak a}
\renewcommand{\b}{\mathfrak b}
\renewcommand{\P}{\mathbb P}
\theoremstyle{definition}
\newtheorem*{thm}{Theorem}
\newtheorem*{prob}{Problem}
\newtheorem*{dfn}{Definition}
\newtheorem*{claim}{Claim}
\theoremstyle{definition}
\newtheorem*{lem}{Lemma}
\newtheorem*{ex}{Exercise}
\newtheorem*{eg}{Example}
\newtheorem*{note}{Note}

\usetikzlibrary{matrix,positioning,quotes}

%==============================================================================
% YOUR DOCUMENT (start here)
%==============================================================================

\begin{document}
\definecolor{myblue}{RGB}{100,180,255}
\definecolor{mygreen}{RGB}{80,160,80}
\definecolor{myred}{RGB}{200,120,100}
\centerline{\LARGE\thishw}


\begin{claim}
For $2\leq a\leq d-1$,
\[
H((d),(a,1,\dots,1),(d-a+1,1,\dots,1))=1
\]
\end{claim}

\begin{claim}
  If $a+b-2=d$, $1\leq n\leq d/2$, and $2\leq a\leq d/2+1$ (equivalently $2\leq a\leq b$), then
\[
H((d-n,n),(a,1,\dots,1),(b,1,\dots,1))=\begin{cases}
\frac 12\min(a-1,n)& d=2n \\
\min(a-1,n)&\text{else}
\end{cases}
\]
\end{claim}
\begin{proof}
  We want to count choices of an $a$-cycle and a $b$-cycle which compose to produce cycle type $(d-n,n)$.
  There are $\binom da$ choices for the points of the $a$-cycle, and the
  points of the $b$-cycle are determined by the two points $x$,$y$ the cycles have in common, of which there are $\binom a2$
  choices.
  Composing these cycles yields cycle type $(n,d-n)$ where $n=n_1+n_2-1$ and
  \[
  n_1=\text{dist}_a(y,x),\quad n_2=\text{dist}_b(x,y)
  \]
  since $y$ can be traced for $n_1-1$ steps along the $a$-cycle to the predecessor of $x$, which then leads to
  the successor of $y$ in the $b$-cycle and another $n_2-1$ steps to reach $x$ again.

  If $n_1$ is fixed, there are $\frac{(a-1)!}{a-1}$ orientations of the $a$-cycle with given $\text{dist}_a(y,x)$ and similarly for the $b$-cycle,
  so our count is
  \[
  \binom da\cdot \binom a2\cdot \frac{(a-1)!}{a-1}\cdot\frac{(b-1)!}{b-1}\sum_{n_1+n_2=n+1}1
  \]
  This yields $\frac 12d!\cdot\min(a-1,n)$, and if $d\neq 2n$ then we multiply by $2$ to account for the case where $n_1+n_2-1=d-n$ instead of $n$.
  \end{proof}

\begin{claim}
  If $a+b-3=d$, $n+m+p=d$, $1\leq n\leq m\leq p$, and $3\leq a\leq \frac{d+3}{2}$ (equivalently $2\leq a\leq b$), then
\begin{align*}
  H((n,m,p),&(a,1,\dots,1),(b,1,\dots,1))\\
  &=\frac 2{|\Aut(n,m,p)|}\cdot \#\{(n_1,m_1):1\leq n_1\leq n,\ \max(1,a-p-n_1)\leq m_1\leq \min(m,a-1-n_1)\}
\end{align*}
In particular if $n=m=1$ and $d>3$, the Hurwitz number is $1$.
\end{claim}
\begin{proof}
  We want to count choices of an $a$-cycle and a $b$-cycle which compose to produce cycle type $(n,m,p)$.
  There are $\binom da$ choices for the points of the $a$-cycle, and the
  points of the $b$-cycle will be determined by the three points $x,y,z$ the cycles have in common. We have $a$ choices for
  $x$ (we will need to divide by $3$ later to account for the fact that $x,y,z$ are unordered).

  If the directionality of the $a$-cycle is $x\to y\to z\to x$ (WLOG), then the same directionality in the $b$-cycle would lead to
  a $d$-cycle, so we will require $x\to z\to y\to x$ in the $b$-cycle to end up with the desired cycle type. In this case
  we have a composed chain \[x\to\cdots\to \text{pred}_a(y)\to\text{succ}_b(y)\cdots\to x\] of length $n_1+n_2-1$ where $n_1=\text{dist}_a(x,y)$ and $n_2=\text{dist}_b(y,x)$, and similarly a composed chain
  \[
  y\to\cdots\to\text{pred}_a(z)\to\text{succ}_b(z)\to\cdots\to y
  \]
  of length $m_1+m_2-1$ where $m_1=\text{dist}_a(y,z)$ and $m_2=\text{dist}_b(z,y)$.

  Choose $n_1,n_2,m_1,m_2$ such that $n_1+n_2-1=n$ and $m_1+m_2-1=m$. Then there are $(a-1)!$ orientations of the $a$-cycle, and $y$ and $z$ are determined by $n_1$ and $m_1$. There are then $(b-3)!$ orientations of the $b$-cycle respecting the gaps $n_2$ and $m_2$. Finally, the number of choices we have in choosing $n,m$ this way out of $n,m,p$ is $\frac{3!}{|\Aut(n,m,p)|}$, so
  \begin{align*}
    H((n,m,p),(a,1,\dots,1),(b,1,\dots,1))&=\binom da\cdot a\cdot\frac 13\cdot\sum_{n_1,n_2,m_1,m_2}(a-1)!(b-3)!\cdot\frac{3!}{|\Aut(n,m,p)|} \\
    &=\frac{d!}{a!(d-a)!}\cdot\frac a3\cdot\frac{(a-1)!(d-a)!\cdot 6}{|\Aut(n,m,p)|}\#\{n_1,n_2,m_1,m_2\}
  \end{align*}
  yielding the result.
  
  \end{proof}

\end{document}
